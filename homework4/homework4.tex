\documentclass[]{article}
\usepackage{lmodern}
\usepackage{amssymb,amsmath}
\usepackage{ifxetex,ifluatex}
\usepackage{fixltx2e} % provides \textsubscript
\ifnum 0\ifxetex 1\fi\ifluatex 1\fi=0 % if pdftex
  \usepackage[T1]{fontenc}
  \usepackage[utf8]{inputenc}
\else % if luatex or xelatex
  \ifxetex
    \usepackage{mathspec}
  \else
    \usepackage{fontspec}
  \fi
  \defaultfontfeatures{Ligatures=TeX,Scale=MatchLowercase}
\fi
% use upquote if available, for straight quotes in verbatim environments
\IfFileExists{upquote.sty}{\usepackage{upquote}}{}
% use microtype if available
\IfFileExists{microtype.sty}{%
\usepackage{microtype}
\UseMicrotypeSet[protrusion]{basicmath} % disable protrusion for tt fonts
}{}
\usepackage[margin=2.25cm]{geometry}
\usepackage{hyperref}
\hypersetup{unicode=true,
            pdftitle={yes},
            pdfborder={0 0 0},
            breaklinks=true}
\urlstyle{same}  % don't use monospace font for urls
\usepackage{color}
\usepackage{fancyvrb}
\newcommand{\VerbBar}{|}
\newcommand{\VERB}{\Verb[commandchars=\\\{\}]}
\DefineVerbatimEnvironment{Highlighting}{Verbatim}{commandchars=\\\{\}}
% Add ',fontsize=\small' for more characters per line
\usepackage{framed}
\definecolor{shadecolor}{RGB}{248,248,248}
\newenvironment{Shaded}{\begin{snugshade}}{\end{snugshade}}
\newcommand{\KeywordTok}[1]{\textcolor[rgb]{0.13,0.29,0.53}{\textbf{#1}}}
\newcommand{\DataTypeTok}[1]{\textcolor[rgb]{0.13,0.29,0.53}{#1}}
\newcommand{\DecValTok}[1]{\textcolor[rgb]{0.00,0.00,0.81}{#1}}
\newcommand{\BaseNTok}[1]{\textcolor[rgb]{0.00,0.00,0.81}{#1}}
\newcommand{\FloatTok}[1]{\textcolor[rgb]{0.00,0.00,0.81}{#1}}
\newcommand{\ConstantTok}[1]{\textcolor[rgb]{0.00,0.00,0.00}{#1}}
\newcommand{\CharTok}[1]{\textcolor[rgb]{0.31,0.60,0.02}{#1}}
\newcommand{\SpecialCharTok}[1]{\textcolor[rgb]{0.00,0.00,0.00}{#1}}
\newcommand{\StringTok}[1]{\textcolor[rgb]{0.31,0.60,0.02}{#1}}
\newcommand{\VerbatimStringTok}[1]{\textcolor[rgb]{0.31,0.60,0.02}{#1}}
\newcommand{\SpecialStringTok}[1]{\textcolor[rgb]{0.31,0.60,0.02}{#1}}
\newcommand{\ImportTok}[1]{#1}
\newcommand{\CommentTok}[1]{\textcolor[rgb]{0.56,0.35,0.01}{\textit{#1}}}
\newcommand{\DocumentationTok}[1]{\textcolor[rgb]{0.56,0.35,0.01}{\textbf{\textit{#1}}}}
\newcommand{\AnnotationTok}[1]{\textcolor[rgb]{0.56,0.35,0.01}{\textbf{\textit{#1}}}}
\newcommand{\CommentVarTok}[1]{\textcolor[rgb]{0.56,0.35,0.01}{\textbf{\textit{#1}}}}
\newcommand{\OtherTok}[1]{\textcolor[rgb]{0.56,0.35,0.01}{#1}}
\newcommand{\FunctionTok}[1]{\textcolor[rgb]{0.00,0.00,0.00}{#1}}
\newcommand{\VariableTok}[1]{\textcolor[rgb]{0.00,0.00,0.00}{#1}}
\newcommand{\ControlFlowTok}[1]{\textcolor[rgb]{0.13,0.29,0.53}{\textbf{#1}}}
\newcommand{\OperatorTok}[1]{\textcolor[rgb]{0.81,0.36,0.00}{\textbf{#1}}}
\newcommand{\BuiltInTok}[1]{#1}
\newcommand{\ExtensionTok}[1]{#1}
\newcommand{\PreprocessorTok}[1]{\textcolor[rgb]{0.56,0.35,0.01}{\textit{#1}}}
\newcommand{\AttributeTok}[1]{\textcolor[rgb]{0.77,0.63,0.00}{#1}}
\newcommand{\RegionMarkerTok}[1]{#1}
\newcommand{\InformationTok}[1]{\textcolor[rgb]{0.56,0.35,0.01}{\textbf{\textit{#1}}}}
\newcommand{\WarningTok}[1]{\textcolor[rgb]{0.56,0.35,0.01}{\textbf{\textit{#1}}}}
\newcommand{\AlertTok}[1]{\textcolor[rgb]{0.94,0.16,0.16}{#1}}
\newcommand{\ErrorTok}[1]{\textcolor[rgb]{0.64,0.00,0.00}{\textbf{#1}}}
\newcommand{\NormalTok}[1]{#1}
\usepackage{longtable,booktabs}
\usepackage{graphicx,grffile}
\makeatletter
\def\maxwidth{\ifdim\Gin@nat@width>\linewidth\linewidth\else\Gin@nat@width\fi}
\def\maxheight{\ifdim\Gin@nat@height>\textheight\textheight\else\Gin@nat@height\fi}
\makeatother
% Scale images if necessary, so that they will not overflow the page
% margins by default, and it is still possible to overwrite the defaults
% using explicit options in \includegraphics[width, height, ...]{}
\setkeys{Gin}{width=\maxwidth,height=\maxheight,keepaspectratio}
\IfFileExists{parskip.sty}{%
\usepackage{parskip}
}{% else
\setlength{\parindent}{0pt}
\setlength{\parskip}{6pt plus 2pt minus 1pt}
}
\setlength{\emergencystretch}{3em}  % prevent overfull lines
\providecommand{\tightlist}{%
  \setlength{\itemsep}{0pt}\setlength{\parskip}{0pt}}
\setcounter{secnumdepth}{0}
% Redefines (sub)paragraphs to behave more like sections
\ifx\paragraph\undefined\else
\let\oldparagraph\paragraph
\renewcommand{\paragraph}[1]{\oldparagraph{#1}\mbox{}}
\fi
\ifx\subparagraph\undefined\else
\let\oldsubparagraph\subparagraph
\renewcommand{\subparagraph}[1]{\oldsubparagraph{#1}\mbox{}}
\fi

%%% Use protect on footnotes to avoid problems with footnotes in titles
\let\rmarkdownfootnote\footnote%
\def\footnote{\protect\rmarkdownfootnote}

%%% Change title format to be more compact
\usepackage{titling}

% Create subtitle command for use in maketitle
\newcommand{\subtitle}[1]{
  \posttitle{
    \begin{center}\large#1\end{center}
    }
}

\setlength{\droptitle}{-2em}
  \title{yes}
  \pretitle{\vspace{\droptitle}\centering\huge}
  \posttitle{\par}
  \author{}
  \preauthor{}\postauthor{}
  \date{}
  \predate{}\postdate{}

\usepackage{../ve414}
\semester{Summer}
\year{2019}
\subtitle{Assignment}
\subtitlenumber{4}
\blockinfo{}
\author{\href{mailto:liuyh615@sjtu.edu.cn}{Yihao Liu} (515370910207)}

\begin{document}
\maketitle

\subsection{Question1}\label{question1}

\subsubsection{(a)}\label{a}

\[f_{Y\mid X}\propto \exp\left[-\frac{(x-y)^2}{2}\right]\times\frac{1}{1+y^2}dy\]
\[\begin{aligned}A&=\int_{-\infty}^\infty f_{X=x\mid Y}f_Y(y)dy \\&= \int_{-\infty}^\infty\exp\left[-\frac{(x-y)^2}{2}\right]\times\frac{1}{1+y^2}dy\\&=\int_{-\pi/2}^{\pi/2}\exp\left[-\frac{(x-\tan u)^2}{2}\right]du\\&\approx \frac{\pi}{n}\sum_{i=1}^n \exp\left[-\frac{(x-\tan u_i)^2}{2}\right].\end{aligned}\]

\[\begin{aligned}E[Y\mid X=x]&=\frac{1}{A}\int_{-\infty}^\infty f_{X=x\mid Y}f_Y(y)ydy\\&=\frac{1}{A}\int_{-\infty}^\infty\exp\left[-\frac{(x-y)^2}{2}\right]\times\frac{y}{1+y^2}dy\\&=\frac{1}{A}\int_{-\pi/2}^{\pi/2}\exp\left[-\frac{(x-\tan u)^2}{2}\right]\tan udu\\&\approx \frac{\pi}{An}\sum_{i=1}^n \exp\left[-\frac{(x-\tan u_i)^2}{2}\right]\tan u_i.\end{aligned}\]

\begin{Shaded}
\begin{Highlighting}[]
\KeywordTok{function}\NormalTok{ grid_approximation(x, n)}
\NormalTok{  y_grid = collect(range(-pi/}\FloatTok{2}\NormalTok{ , length=n , stop=pi/}\FloatTok{2}\NormalTok{))}
\NormalTok{  unnormalised_posterior = map(u->exp(-(x-tan(u))^}\FloatTok{2}\NormalTok{/}\FloatTok{2}\NormalTok{), y_grid)}
\NormalTok{  unnormalised_expectation = map(u->exp(-(x-tan(u))^}\FloatTok{2}\NormalTok{/}\FloatTok{2}\NormalTok{)*tan(u), y_grid)}
\NormalTok{  A = pi * sum(unnormalised_posterior) / n}
\NormalTok{  E = pi * sum(unnormalised_expectation) / A / n}
  \KeywordTok{return}\NormalTok{ A, E}
\KeywordTok{end}
\NormalTok{grid_sizes = [}\FloatTok{50}\NormalTok{, }\FloatTok{250}\NormalTok{, }\FloatTok{750}\NormalTok{, }\FloatTok{1500}\NormalTok{, }\FloatTok{3000}\NormalTok{]}
\KeywordTok{for}\NormalTok{ n }\KeywordTok{in}\NormalTok{ grid_sizes}
\NormalTok{  println(grid_approximation(}\FloatTok{0.5}\NormalTok{, n))}
\KeywordTok{end}
\end{Highlighting}
\end{Shaded}

\subsubsection{(b)}\label{b}

\begin{longtable}[]{@{}ccc@{}}
\toprule
Grid size \(n\) & \(A\) & \(E[Y\mid X=0.5]\)\tabularnewline
\midrule
\endhead
50 & 1.518533686 & 0.2661760387\tabularnewline
250 & 1.543326174 & 0.2661761234\tabularnewline
750 & 1.547458239 & 0.2661761234\tabularnewline
1000 & 1.548491255 & 0.2661761234\tabularnewline
3000 & 1.549007763 & 0.2661761234\tabularnewline
\bottomrule
\end{longtable}

\subsubsection{(c)}\label{c}

\begin{Shaded}
\begin{Highlighting}[]
\NormalTok{using Distributions}
\KeywordTok{function}\NormalTok{ direct_grid_approximation(x, n, m)}
\NormalTok{  y_grid = collect(range(-pi/}\FloatTok{2}\NormalTok{ , length=n , stop=pi/}\FloatTok{2}\NormalTok{))}
\NormalTok{  unnormalised_posterior = map(y->exp(-(x-tan(y))^}\FloatTok{2}\NormalTok{/}\FloatTok{2}\NormalTok{), y_grid)}
\NormalTok{  A = pi * sum(unnormalised_posterior) / n}
\NormalTok{  posterior = unnormalised_posterior / A}
\NormalTok{  samples = map(y->tan(y), wsample(y_grid, posterior, m))}
\NormalTok{  E = sum(samples) / m}
  \KeywordTok{return}\NormalTok{ A, E}
\KeywordTok{end}
\NormalTok{grid_sizes = [}\FloatTok{50}\NormalTok{, }\FloatTok{250}\NormalTok{, }\FloatTok{750}\NormalTok{, }\FloatTok{1500}\NormalTok{, }\FloatTok{3000}\NormalTok{]}
\NormalTok{sample_sizes = [}\FloatTok{100}\NormalTok{, }\FloatTok{1000}\NormalTok{]}
\KeywordTok{for}\NormalTok{ n }\KeywordTok{in}\NormalTok{ grid_sizes}
  \KeywordTok{for}\NormalTok{ m }\KeywordTok{in}\NormalTok{ sample_sizes}
\NormalTok{    println(direct_grid_approximation(}\FloatTok{0.5}\NormalTok{, n, m))}
  \KeywordTok{end}
\KeywordTok{end}
\end{Highlighting}
\end{Shaded}

\subsubsection{(d)}\label{d}

\begin{longtable}[]{@{}cccc@{}}
\toprule
Grid size \(n\) & Sample size \(m\) & \(A\) &
\(E[Y\mid X=0.5]\)\tabularnewline
\midrule
\endhead
50 & 100 & 1.518533686 & 0.2209137719\tabularnewline
50 & 1000 & 1.518533686 & 0.2768561719\tabularnewline
250 & 100 & 1.543326174 & 0.2504343983\tabularnewline
250 & 1000 & 1.543326174 & 0.2873817359\tabularnewline
750 & 100 & 1.547458239 & 0.2189899855\tabularnewline
750 & 1000 & 1.547458239 & 0.266422273\tabularnewline
1000 & 100 & 1.548491255 & 0.3766259741\tabularnewline
1000 & 1000 & 1.548491255 & 0.2669810206\tabularnewline
3000 & 100 & 1.549007763 & 0.2906436677\tabularnewline
3000 & 1000 & 1.549007763 & 0.2927511428\tabularnewline
\bottomrule
\end{longtable}

\subsubsection{(e)}\label{e}

\begin{Shaded}
\begin{Highlighting}[]
\KeywordTok{function}\NormalTok{ no_transform_grid_approximation(x, a, b, n) }
  \KeywordTok{if}\NormalTok{ n <= }\FloatTok{1000}
\NormalTok{    y_grid = collect(range(a, length=n, stop=b))}
\NormalTok{    newa = a}
  \KeywordTok{elseif}\NormalTok{ n > }\FloatTok{1000}\NormalTok{ && n <= }\FloatTok{2000}
\NormalTok{    nm = }\FloatTok{1000}
\NormalTok{    na = round(}\DataTypeTok{Int}\NormalTok{, (n-nm)/}\FloatTok{2}\NormalTok{)}
\NormalTok{    l = (b-a)/(nm-}\FloatTok{1}\NormalTok{)}
\NormalTok{    newa = a - l*na}
\NormalTok{    y_grid = collect(range(newa, step=l, length=n))}
  \KeywordTok{else}\NormalTok{ n > }\FloatTok{2000}
\NormalTok{    nm = round(}\DataTypeTok{Int}\NormalTok{, n/}\FloatTok{2}\NormalTok{)}
\NormalTok{    na = round(}\DataTypeTok{Int}\NormalTok{, (n-nm)/}\FloatTok{2}\NormalTok{)}
\NormalTok{    l = (b-a)/(nm-}\FloatTok{1}\NormalTok{)}
\NormalTok{    newa = a - l*na}
\NormalTok{    y_grid = collect(range(newa, step=l, length=n))}
  \KeywordTok{end}
\NormalTok{  unnormalised_posterior = map(y->exp(-(x-y)^}\FloatTok{2}\NormalTok{/}\FloatTok{2}\NormalTok{)/(}\FloatTok{1}\NormalTok{+y^}\FloatTok{2}\NormalTok{), y_grid)}
\NormalTok{  unnormalised_expectation = map(y->exp(-(x-y)^}\FloatTok{2}\NormalTok{/}\FloatTok{2}\NormalTok{)/(}\FloatTok{1}\NormalTok{+y^}\FloatTok{2}\NormalTok{)*y, y_grid)}
\NormalTok{  A = (y_grid[n]-y_grid[}\FloatTok{1}\NormalTok{]) * sum(unnormalised_posterior) / n}
\NormalTok{  E = (y_grid[n]-y_grid[}\FloatTok{1}\NormalTok{]) * sum(unnormalised_expectation) / A / n}
  \KeywordTok{return}\NormalTok{ A, E}
\KeywordTok{end}

\NormalTok{grid_sizes = [}\FloatTok{50}\NormalTok{, }\FloatTok{250}\NormalTok{, }\FloatTok{750}\NormalTok{, }\FloatTok{1500}\NormalTok{, }\FloatTok{3000}\NormalTok{]}
\KeywordTok{for}\NormalTok{ n }\KeywordTok{in}\NormalTok{ grid_sizes}
\NormalTok{  println(no_transform_grid_approximation(}\FloatTok{0.5}\NormalTok{, -}\FloatTok{5}\NormalTok{, }\FloatTok{5}\NormalTok{, n))}
\KeywordTok{end}
\end{Highlighting}
\end{Shaded}

\begin{longtable}[]{@{}ccc@{}}
\toprule
Grid size \(n\) & \(A\) & \(E[Y\mid X=0.5]\)\tabularnewline
\midrule
\endhead
50 & 1.518533615 & 0.2661755641\tabularnewline
250 & 1.543325899 & 0.2661752529\tabularnewline
750 & 1.547457943 & 0.2661751929\tabularnewline
1000 & 1.548491255 & 0.2661761234\tabularnewline
3000 & 1.549007763 & 0.2661761234\tabularnewline
\bottomrule
\end{longtable}

We can find that the results are similar, especially when $n$ is large. It is because the probability when $|y|>5$ is very small so that the two grid approximation can be considered the same when $|y|$ becomes larger.


\end{document}
